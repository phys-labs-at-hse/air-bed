\documentclass[a4paper, 12pt]{article}

\usepackage[no-math]{fontspec-xetex}
\setmainfont{IBM Plex Sans}
\usepackage[english, russian]{babel}

\usepackage{blindtext}
\usepackage{microtype}
\usepackage{geometry}

\usepackage{amsmath, amsfonts, amssymb, amsthm, mathtools}
\usepackage{MnSymbol}
\usepackage{physics}

\usepackage{graphicx, wrapfig, caption, subcaption}
\usepackage{color, xcolor}

\usepackage{titlesec}
\usepackage[document]{ragged2e}
\usepackage{enumitem}
\usepackage{hyperref}
\usepackage{import}

\graphicspath{{figures/}}
\geometry{margin=6em}
\geometry{bottom=6em}

% Настройки заголовков
\titleformat{\section}[hang]{\Large}{}{0em}{}{}
\titleformat{\subsection}[hang]{\large}{}{0em}{}{}

% Настройки параграфов текста
\setlength{\parindent}{0em}
\setlength{\parskip}{1em}
\renewcommand{\baselinestretch}{1.1}

% Настройки списков
\setlist{leftmargin=*, noitemsep}

% Настройки таблиц
\setlength{\tabcolsep}{2em}
\renewcommand{\arraystretch}{1.3}

% Настройки ссылок
\hypersetup{colorlinks=true, linkcolor=blue, urlcolor=blue}

\title{Вращение на водушной подушке}
\author{Роман Ухоботов, Николай Грузинов}
\date{собрано \today}

\begin{document}
\maketitle

\section{Используемое оборудование}
\begin{enumerate}
\item грузы, из которых можно сделать грузы массами 1, 2, 3, 4 грамма;
\item вращающаяся на воздушной подушке платформа;
\item грузы для увеличения момента инерции (3 пары: маленькие, средние, большие)
\item весы, штангенциркуль;
\item камера телефона.
\end{enumerate}

\section{Цели и задачи}
Цель: изучить, как ускоряется платформа в зависимости от того, какие грузы вешать (для увеличение её момента инерции или увеличения её силы).
Задачи:
\begin{enumerate}
\item изучить торможение платформы при свободном вращении;
\item проверить, является ли движение платформы равноускоренным;
\item проверить простые соотношения между ускорением платформы, моментом силы, и моментом инерции.
\end{enumerate}

\section{Теоретическая модель}

\subsection{Без трения}
Если за нитку подвешен груз массы $m$, и нитка намотана на бобину радиуса $r$ на платформе, то, пренебрегая трением, можно записать:
\[ I \beta = mgr .\]

\subsection{Про трение}
Разумно предположить, что если трение есть, то оно возникает из-за сопротивления воздуха.
Малейшие касания платформы приводили к застреваниям, поэтому сухое трение мы бы сразу заметили.
Тогда, в первом приближении, момент, создаваемый силой трения, должен быть пропорционален скорости платформы.

\section{Методика измерений}
Мы подвешивали груз определенной массы, и снимали вращение платформы на камеру.
Из видео далее извлекалась зависимость угла поворота от времени, иногда не с самого начала вращения, по техническим причинам.

\section{Результаты}
\section{Выводы}
\end{document}
